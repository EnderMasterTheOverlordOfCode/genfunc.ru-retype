\documentclass[12pt, letterpaper]{extarticle}
\usepackage[utf8]{inputenc}
\usepackage[T2A,T1]{fontenc}
\usepackage[english, russian]{babel}
\usepackage{amsmath}
\usepackage{amsfonts}
\usepackage{hyperref}
\hypersetup{colorlinks=true,linkcolor=blue,filecolor=magenta,urlcolor=cyan,}
\pagenumbering{gobble}
\usepackage[left=1cm, top=1cm, right=1cm, bottom=1cm, nohead, nofoot]{geometry}
\title{\textbf{Преобразование производящих функций}}
\author{\href{https://web.archive.org/web/20170629072301/http://www.genfunc.ru/theory/oper/}{genfunc.ru(18.03.2018)}}
\date{давным-давно в далёкой галактике}

\everymath{\displaystyle}

\begin{document}
\maketitle
При решении рекуррентных соотношений были использованы некоторые приёмы работы с производящими функциями. Здесь эти приёмы рассмотрены более подробно и оформлены в виде таблицы в конце раздела.
\begin{center}
  \textbf{ОПРЕДЕЛЕНИЕ}
\end{center}

Напоминаю, что мы находимся в рамках определения, данного во <<Введении>>, а именно: \textbf{производящая функция последовательности} $(a0, a1, a2, \dotsc)$ есть формальный степенной ряд
\[G(z)=\sum_{n=0}^{\infty} a_nz^n = a_0+a_1z+a_2z^2+\dotsb\]

Говорят, что функция <<производит>>, или <<генерирует>> последовательность, так как в разложении функции $G(z)$ ряд по степеням $z$, коэффициент при $z_n$ равен $a_n$. Обозначается этот факт с помощью записи:
\[[z^n]G(z) = a_n\]

Данная запись так и читается: <<коэффициент при $z_n$ в разложении функции $G$ по степеням $z$>>.

Последовательность $(a_n)$ начинается от нуля, но иногда удобно считать, что $n$ — произвольное целое число. Для этого полагаем $a_{-1}=a_{-2}=\dotsb=0$.

Производящая функция является \textbf{формальным рядом}, то есть не следует считать, что $z$ — число и пытаться строгость рассуждений привязывать к доказательству сходимости степенных рядов. Мы смотрим на производящие функции как на формальные суммы, не принимая во внимание их сходимость. Существуют ряды, которые расходятся во всех точках кроме нуля, например,
\[G(z) = \sum_{n=0}^{\infty}n!z^n = 1+ z+ 2z^2 + 6z^3+24z^4+120z^5+\dotsb\]
однако манипуляции с ними остаются корректными все равно. Строгое доказательство преобразований формальных степенных рядов здесь не приводится.

В любом случае, если Вы не доверяете механизму производящих функций, можно поступить следующим образом: получить некоторую формулу, а затем доказать её строго другим способом, например, методом математической индукции. Скажем, получили формулу для чисел Фибоначчи -- проверьте, что она удовлетворяет рекуррентному соотношению и выполняются начальные условия. Проще доказать формулу, которая уже у нас в руках, чем выводить её <<с нуля>>.
\newpage
\begin{center}
  \textbf{СДВИГ}
\end{center}

Итак, функция $G(z)$ генерирует последовательность $(a_0, a_1, a_2, \dotsc)$. Что даёт умножение всей функции на $z_m$ ($m\geq0$, целое)? Для ответа на этот вопрос распишем процесс умножения подробнее:
\[z^mG(z)=z^m\sum_{n=0}^{\infty} a_nz^n = a_0z^m+a_1z^{m+1}+a_2z^{m+2}+\dotsb\]
Данная функция генерирует новую последовательность:
\[(\underbrace{0,0,\dotsc,0}_m,a_0, a_1, \dotsc)\]
которая является сдвигом исходной последовательности на m элементов вправо. То есть
\[[z^n]\Bigl( z^mG(z) \Bigr) = a_{n-m}\]
Обратите внимание на то, что запись корректна, поскольку при $n<m$ получаются отрицательные индексы, а элементы с такими индексами мы договорились считать нулями.

Аналогично, но в обратную сторону, происходит процесс деления. Для того, чтобы отрицательные (после сдвига) члены были равны нулю, нужно вычесть первые $m$ слагаемых:
\[\frac{G(z)-a_0-a_1z-\dotsb-a_{m-1}z^{m-1}}{z^m} = a_m+a_{m+1}z^{}+a_{m+2}{z^{2}}+\dotsb\]
То есть
\[[z^n]\Biggl( \frac{G(z)-a_0-a_1z-\dotsb-a_{m-1}z^{m-1}}{z^m} \Biggr) = a_{n+m}\]
\begin{center}
  \textbf{ПОЛИНОМИАЛЬНЫЙ МНОЖИТЕЛЬ И ДЕЛИТЕЛЬ}
\end{center}

Довольно часто приходится иметь дело с последовательностью $(na_n)$. Такая последовательность получается путём дифференцирования функции $G(z)$ с последующим умножением результата на $z$:
\[zG'(z) = z\left(\sum_{n=0}^{\infty}a_nz^n\right)' = z\sum_{n=1}^{\infty}na_nz^{n-1} = \sum_{n=0}^{\infty}na_nz^{n}\]

В предпоследней сумме индекс суммирования стал начинаться от $1$, так как производная константы $a'0=0$, а в последней сумме мы снова заставили его начинаться от $0$, так как при $n=0$ величина $(0\cdot a_0z^0)$ все равно равна нулю и не влияет на сумму. Таким образом,
\[[z^n]\Bigl( zG(z) \Bigr) = na_{n}\]

Аналогично, интегрирование функции $\frac{G(z)-a0}{z}$ поделит общий член последовательности на $n$ (заметьте, что мы предусмотрительно сдвинули элементы влево, а иначе у нас получилась бы последовательность $\frac{a_{n-1}}{n}n\geq1$ [проверьте]).
\[\int\limits_0^z \frac{G(t)-a_0}{t} dt = \int\limits_0^z\sum_{n=1}^{\infty}\left(a_nt^{n-1}\right)dt=\sum_{n=1}^{\infty}\frac{a_n}{n}z^{n}\]

Обратите внимание, что в новой последовательности нулевой член равен нулю.
\[[z^n]\left(\int\limits_0^z\frac{G(t)-a_0}{t}dt\right)=\left\{\begin{array}{rcl}n=0&:&0 \\[7pt] n>0&:&\frac{a_n}{n}\end{array}\right.\]
\newpage
\begin{center}
  \textbf{СЛОЖЕНИЕ И УМНОЖЕНИЕ НА КОНСТАНТУ}
\end{center}

Рассмотрим две производящие функции:
\[G(z)=\sum_{n=0}^{\infty}a_nz^n=a_0+a_1z+a_2z^2+\dotsb\]
\[H(z)=\sum_{n=0}^{\infty}b_nz^n=b_0+b_1z+b_2z^2+\dotsb\]
Мы можем их формально сложить:
\[G (z) + H(z) = a_0+b_0+(a_1+b_1)z+( a_2+b_2 )z^2+ \cdots = \sum_{n=0}^{\infty}(a_n+b_n)z^n\]
что, в общем-то не удивительно.

Производящую функцию можно умножить на постоянный множитель:
\[\alpha G ( z ) = \alpha a_0+\alpha a_1 z+ \alpha a_2 z^2 + \dotsb = \sum_{n=0}^{\infty}\alpha a_nz^n\]

Символ $z$ также можно умножить на постоянное число:

\[G ( rz ) = a_0+ a_1 rz+ a_2 (rz)^2 + \dotsb = \sum_{n=0}^{\infty}r^n a_n z^n\]
Таким образом,
\[
  \begin{array}{l c l}
    [z^n]\left(G(z)+H(z)\right)&=&a_n+b_n\\{}
    [z^n]\left(\alpha G(z)\right)&=&\alpha a_n\\{}
    [z^n]\left(G(rz)\right)&=&r^na_n
  \end{array}
\]
\begin{center}
  \textbf{СВЁРТКА}
\end{center}

Перемножим производящие функции $G(z)$ и $H(z)$:

\[F = G\cdot H = a_0b_0+(a_0b_1+a_1b_0)z+( a_0b_2 + a_1b_1+a_2b_0)z^2+\dotsb=\sum_{n=0}^{\infty}\Biggl( \sum_{k=0}^n a_kb_{n-k} \Biggr)z^n\]
Сумму, взятую в скобки, принято называть <<свёрткой>> (обозначим её через $c_n$):
\[c_n = \sum_{k=0}^n a_kb_{n-k}\]

Таким образом, произведение производящих функций, генерирующих последовательности $(a_n)$ и $(b_n)$, генерирует свёртку $(c_n)$.

Рассмотрим частный случай этого замечательного факта. Пусть $b_n=1$, тогда
\[F(z) = \frac{1}{1-z}G(z) = a_0 + (a_0+a_1)z + (a_0+a_1+a_2)z^2+\dotsb = \sum_{n=0}^{\infty}\Biggl( \sum_{k=0}^n a_k \Biggr)z^n\]

То есть функция $\frac{G(z)}{1-z}$ генерирует последовательность частичных сумм исходной последовательности $(a_n)$. Такое положение дел даёт нам много свободы, например, теперь легко представлять себе разложение в ряд следующих дробей, не вспоминая явно про биномиальные коэффициенты:

\[G(z) = \frac{1}{(1-x)^2}\]
\[a_n = \sum_{i=0}^{n-1}1=n\]
\[G(z) = \sum_{n=0}^{\infty}nz^n\]
Обратите внимание на то, как следующая последовательность получается из частичных сумм предыдущей.

Далее, используя свёртку, можно доказать некоторые полезные тождества. Мы знаем, чему равна сумма биномиальных коэффициентов с целым положительным верхним индексом $n$: Она равна $2^n$. А чему равна сумма их квадратов? — $\sum_{k=0}^{n} \binom{n}{k}^2 =\; \text{?}, \quad n\geq 0, \text{ целое}.$

Распишем сумму подробнее:
\[\sum_{k=0}^{n} \binom{n}{k}\binom{n}{k} = \sum_{k=0}^{n} \binom{n}{k}\binom{n}{n-k}\]
Теперь хорошо видно, что это свёртка двух последовательностей, каждая из которых порождается функцией
\[(1+z)^n = \sum_{k=0}^{n} \binom{n}{k}\]
Значит
\[\sum_{k=0}^{n} \binom{n}{k}^2 = [z^n](1+z)^{2n} = \binom{2n}{n}, \quad n\geq 0, \text { целое}.\]

Кстати, эта красивая последовательность биномиальных коэффициентов из 2n по n украшает верхнюю часть нашего сайта.
\begin{center}
  \textbf{ЧЁТНЫЕ И НЕЧЁТНЫЕ ЭЛЕМЕНТЫ}
\end{center}

Рассмотрим сумму
\[G(z) + G(-z) = \sum_{n=0}^{\infty} (a_nz^n+a_n(-z)^n) = a_n(1+(-1)^n)z^n=2\sum_{n=0}^{\infty} a_{2n}z^{2n}\]
Таким образом, производящая функция $\frac{G(z)+G(-z)}{2}$ генерирует последовательность \\$(a_0, 0, a_2, 0, a_4, \dotsc)$, то есть последовательность, в которой элементы с нечётными номерами заменены нулями. Например, пусть $G(z)=\frac{1}{1-z}$, тогда

\[\frac{G(z)+G(-z)}{2} = \frac12\left(\frac{1}{1-z}+\frac{1}{1+z}\right) = \frac{1}{1-z^2}\]

Полученная производящая функция генерирует последовательность $(1, 0, 1, 0,\dotsc)$, что согласуется с таблицей производящих функций (последовательность №5 при $m=2$).

Последовательность, в которой элементы с нечётным индексом вырезаны, можно <<уплотнить>> заменой $z$ на $z^1/2$:
\[\frac{G(\sqrt{z})+G(-\sqrt{z})}{2} = \sum_{n=0}^{\infty} a_{2n}z^n\]
Такая производящая функция генерирует последовательность из элементов с чётными номерами в исходной последовательности: $(a_0, a_2, a_4, \dotsc)$.

\[\text{Рассуждая аналогично, получим:}\quad\frac{G(z)-G(-z)}{2} = \sum_{n=0}^{\infty} a_{2n+1}z^{2n+1}\]

Такая функция <<обнуляет>> элементы, стоящие на чётных позициях. Вопрос: как <<уплотнить>> такую последовательность (убрать появившиеся нули)?
\begin{center}
  \textbf{ТАБЛИЦА ПРЕОБРАЗОВАНИЙ}
\end{center}

Предполагается, что
\[G(z) = \sum_{n=0}^{\infty}a_nz^n=a_0+a_1z+a_2z^2+\dotsb\]
\[H(z) = \sum_{n=0}^{\infty}b_nz^n=b_0+b_1z+b_2z^2+\dotsb\]

В таблице указаны основные преобразования, которые можно законно выполнять с производящими функциями $G(z)$ и $H(z)$, а также комплексными числами $\alpha$, $\beta$, $\gamma$ и целым $m$. В левой колонке указано действие над функциями, а в правой — порождаемая после этого действия последовательность.
\[\text{А здесь я ничего не учидел (предложения принимаются)}\]
\end{document}
