\documentclass[12pt, letterpaper]{extarticle}
\usepackage[utf8]{inputenc}
\usepackage[T2A,T1]{fontenc}
\usepackage[english, russian]{babel}
\usepackage{amsmath}
\usepackage{amsfonts}
\usepackage{hyperref}
\hypersetup{colorlinks=true,linkcolor=blue,filecolor=magenta,urlcolor=cyan,}
\pagenumbering{gobble}
\usepackage[left=1cm, top=1cm, right=1cm, bottom=1cm, nohead, nofoot]{geometry}
\usepackage{titlesec}
\title{\textbf{Задача о счастливых билетах}}
\author{\href{https://web.archive.org/web/20170629072301/http://www.genfunc.ru/theory/oper/}{genfunc.ru(18.03.2018)}}
\date{давным-давно в далёкой галактике}
\titleformat{\section}{\LARGE\bfseries\filcenter}{}{1em}{}
\titleformat{\subsection}{\large\bfseries\filcenter}{}{1em}{}

\everymath{\displaystyle}

\begin{document}
\maketitle
Одним из классических примеров использования производящих функций является задача о счастливых билетах.

Троллейбусный (трамвайный) билет имеет номер, состоящий из шести цифр. Билет считается счастливым, если сумма первых трёх цифр равна сумме последних трёх, например, $024321$. Первая цифра номера билета может быть нулём. Известно, что количество счастливых билетов из шести цифр равно $55252$. Но как это число было получено? Вообще, как решать более сложную задачу: для любого положительного целого $n$ указать количество $2n$-значных счастливых билетов?

Здесь будут рассмотрены некоторые известные методы решения данной задачи. Количество счастливых билетов из $2n$ цифр будем обозначать символом $L_n$.
\section{МЕТОД ДИНАМИЧЕСКОГО ПРОГРАММИРОВАНИЯ}

Введём обозначение: $D_n^k$ — количество $n$-значных чисел с суммой цифр, равной $k$ (число может начинаться с цифры $0$). Понятно, что любой билет состоит из двух частей: левой ($n$ цифр) и правой (тоже $n$ цифр), причём в обеих частях сумма цифр одинакова. Количество счастливых билетов с суммой $k$ в одной из частей, очевидно, равно $(D_n^k)^2$. Значит общее количество $2n$-значных счастливых билетов равно
\[L_n = \sum_{k=0}^{9n} (D_{n}^{k})^2\]
Верхний индекс суммирования равен $9n$, так как максимальная сумма цифр в одной части билета равна $9n$.

Теперь осталось найти все значения $D_n^k$. Количество $n$-значных чисел с суммой цифр $k$ можно выразить через количество ($n-1$)-значных чисел, добавляя к ним $n$-ю цифру, которая может быть равна $0, 1, \dotsc, 9$:
\[D_n^k = \sum_{j=0}^{9} D_{n-1}^{k-j} \qquad n>0\]
Здесь неявно предполагается, что $D_n^{-1}=D_n^{-2}=\dotsb=0$ для $n\geq0$. Положим по определению $D_0^0=1$.

Вычисление значений $D_n^k$ по указанной формуле лучше представить с помощью таблицы($n$ -- от $0$ до $3$, $k$ -- от $0$ до $27$):

\[
  \begin{array}{|r| r r r r}
    \hline
    D_n^k & 0 & 1 & 2 & 3 \\
    \hline
    0 & 1 & 1 & 1 & 1 \\
    1 &   & 1 & 2 & 3 \\
    2 &   & 1 & 3 & 6 \\
    3 &   & 1 & \color{blue}{4} & 10 \\
    4 &   & 1 & \color{blue}{5} & 15 \\
    5 &   & 1 & \color{blue}{6} & 21 \\
    6 &   & 1 & \color{blue}{7} & 28 \\
    7 &   & 1 & \color{blue}{8} & 36 \\
    8 &   & 1 & \color{blue}{9} & 45 \\
    9 &   & 1 & \color{blue}{10} & 55 \\
    10 &   &   & \color{blue}{9} & 63 \\
    11 &   &   & \color{blue}{8} & 69 \\
    12 &   &   & \color{blue}{7} & \color{red}{73} \\
    13 &   &   & 6 & 75 \\
    14 &   &   & 5 & 75 \\
    15 &   &   & 4 & 73 \\
    16 &   &   & 3 & 69 \\
    17 &   &   & 2 & 63 \\
    18 &   &   & 1 & 55 \\
    19 &   &   &   & 45 \\
    20 &   &   &   & 36 \\
    21 &   &   &   & 28 \\
    22 &   &   &   & 21 \\
    23 &   &   &   & 15 \\
    24 &   &   &   & 10 \\
    25 &   &   &   & 6 \\
    26 &   &   &   & 3 \\
    27 &   &   &   & 1 \\
  \end{array}
\]

Любое число в этой таблице (кроме $D_0^0$) получается если просуммировать $10$ элементов, стоящих слева и сверху от него. Например, в таблице красным цветом выделено число $73$, а синим — числа, сумме которых оно равно. Само это число $73$ означает, что именно столько существует трёхзначных чисел с суммой цифр $12$.

Теперь нужно просуммировать квадраты чисел, стоящих в столбце $n=3: 12+32+62 +\dotsb=55252$. Если нужно было бы подсчитать восьмизначные билеты, то потребовалось бы вычислять столбец $n=4$ до значения $k=36$.
\section{МЕТОД ПРОИЗВОДЯЩИХ ФУНКЦИЙ}

Билет состоит из двух частей. Рассмотрим произвольный счастливый билет, скажем, $271334$ и заменим цифры второй его части на величину, которой им не хватает до $9$. То есть $271665$. Теперь сумма всех цифр билета равна $27$. Легко заметить, что такой фокус проходит с любым счастливым билетом. Таким образом, количество счастливых билетов из $2n$ цифр равно количеству $2n$-значных чисел с суммой цифр, равной $9n$. То есть

\[
  L_n = D_{2n}^{9n}.
\]

Теперь можно было бы воспользоваться техникой предыдущего пункта и найти число, стоящее в столбце $n=6$ и в строке $k=27$. Получилось бы в точности $55252$. Но здесь можно воспользоваться техникой производящих функций.

Выпишем производящую функцию $G(z)$, коэффициент при $z^k$ у которой будет равен $D_1^k$:

\[
  G(z) = 1+z+z^2+z^3+z^4+z^5+z^6+z^7+z^8+z^9
\]
Действительно, однозначное число с суммой цифр $k$ (для $k=0,\dotsc,9$) можно представить одним способом. Для $k>9$ — ноль способов.

Заметим, что если возвести функцию $G$ в квадрат, то коэффициент при $z^k$ будет равен числу способов получить сумму $k$ с помощью двух цифр от $0$ до $9$:

\[
  G^2(z) = 1+2z^2+3z^2+4z^3+5z^4+6z^5+6z^{13}+5z^{14}+4z^{15}+3z^{16}+2z^{17}+z^{18}
\]

В общем случае, $G^n(z)$ — это производящая функция для чисел $D_n^k$, поскольку коэффициент при $z^k$ получается перебором всех возможных комбинаций из $n$ цифр от $0$ до $9$, равных в сумме $k$. Перепишем производящую функцию в ином виде:

\[
  G(z) = 1+z+\dotsb+z^9 = \frac{1-z^{10}}{1-z}
\]

В итоге, нам требуется отыскать
\[
  [z^{27}]G^6(z).
\]

Для этого посмотрим, что будет получаться, если раскрывать скобки в следующем выражении (нас интересует только коэффициенты при $z^{27}$):

\begin{gather*}
  G^6(z)=(1-z^{10})^6(1-z)^{-6}=\sum_{k=0}^{\infty}{\binom{6}{k}(-z^{10})^k}\sum_{k=0}^{\infty}{\binom{-6}{k}(-z)^k}=\\
 \dotsb +\left(-\binom{6}{0}\binom{-6}{27}+\binom{6}{1}\binom{-6}{17}-\binom{6}{2}\binom{-6}{7}\right)z^{27}+\dotsb
\end{gather*}

Таким образом,

\[
  D_6^{27}=
  -\binom{6}{0}\binom{-6}{27}+\binom{6}{1}\binom{-6}{17}-\binom{6}{2}\binom{-6}{7}=\binom{32}{5}-6\binom{22}{5}+15\binom{12}{5}=55252
\]

\section{РЕШЕНИЕ ПУТЁМ ИНТЕГРИРОВАНИЯ}

Внимание, данный раздел предназначен для тех, кто знаком с курсом ТФКП.

Воспользуемся производящей функцией $G(z)$ из предыдущего раздела:

\[
G(z) = 1+z+z^2+z^3+z^4+z^5+z^6+z^7+z^8+z^9 = \frac{1-z^{10}}{1-z}
\]
Составим ряд Лорана следующим образом:

\[
  H(z)=G^3(z)G^3(1/z)=\sum_{k=-\infty}^{\infty}a_kz^k
\]

Значение $a_0$ в данном разложении будет в точности равно [ проверьте ]

\[
a_0=\bigl(D_3^0\bigr)^2+\bigl(D_3^1\bigr)^2+\cdots+\bigl(D_3^{27}\bigr)^2=L_3
\]

Интегральная теорема Коши говорит, что

\[
  [z^0]H(z) = \frac{1}{2\pi i} \oint \frac{H(z)}{z} dz
\]
где интегрирование выполняется по любой простой замкнутой кривой, охватывающей начало координат. Удобно взять $z=e^{i\varphi }$, чтобы интегрировать вдоль окружности (такая замена равносильна переходу к полярным координатам):

\[
  [z^0]H(e^{i\varphi })=
  \frac{1}{2\pi i}\int\limits_0^{2\pi}{e^{-i\varphi}H(e^{i\varphi})}de^{i\varphi}=
  \frac{1}{2\pi}\int\limits_0^{2\pi}{H(e^{i\varphi})}d\varphi
\]

Теперь подставим сюда $H(z)$:

\begin{gather*}
  a_0=
  \frac{1}{2\pi}\int\limits_0^{2\pi}{\frac{(1e-^{10i\varphi})^3(1-e^{-10i\varphi})^3}{(1-e^{i\varphi})^3(1-e^{-i\varphi})^3}}d\varphi=
  \frac{1}{2\pi}\int\limits_0^{2\pi}{\frac{(2-e^{10i\varphi}-e^{-10i\varphi})^3}{(2-e^{i\varphi}-e^{-i\varphi})^3}}d\varphi=\\
  \frac{1}{2\pi}\int\limits_0^{2\pi}{\frac{(1-\cos{10\varphi})^3}{(1-\cos{\varphi})^3}}d\varphi=
  \frac{1}{2\pi}\int\limits_0^{2\pi}{\left(\frac{\sin{5\varphi}}{\sin{(\varphi/2)}}\right)^6}d\varphi=
  \frac{1}{\pi}\int\limits_0^{\pi}{\left(\frac{\sin{10\varphi}}{\sin{\varphi}}\right)^6}d\varphi
\end{gather*}
Итак,

\[
  L_3=\frac{1}{\pi}\int\limits_{0}^{\pi}{\left(\frac{\sin10\varphi }{\sin\varphi }\right)^6}d\varphi =55252
\]

Нетрудно видеть, что в общем случае

\[
L_n=\frac{1}{\pi}\int\limits_{0}^{\pi}{\left(\frac{\sin10\varphi}{\sin\varphi}\right)^{2n}}d\varphi .
\]

Отметим, что в реальности решать эту задачу путём интегрирования чрезвычайно затруднительно. На практике лучше работает метод динамического программирования.

Также в одном из упражнений будет предложено вывести формулу:

\[
  L_n = \sum_{k=0}^{n-1} (-1)^k \binom{2n}{k}\binom{11n-1-10k}{2n-1}
\]
вычисление по которой представляется на сегодняшний день очень эффективным.
\end{document}
